%% Generated by Sphinx.
\def\sphinxdocclass{report}
\documentclass[letterpaper,10pt,english]{sphinxmanual}
\ifdefined\pdfpxdimen
   \let\sphinxpxdimen\pdfpxdimen\else\newdimen\sphinxpxdimen
\fi \sphinxpxdimen=.75bp\relax
\ifdefined\pdfimageresolution
    \pdfimageresolution= \numexpr \dimexpr1in\relax/\sphinxpxdimen\relax
\fi
%% let collapsible pdf bookmarks panel have high depth per default
\PassOptionsToPackage{bookmarksdepth=5}{hyperref}

\PassOptionsToPackage{booktabs}{sphinx}
\PassOptionsToPackage{colorrows}{sphinx}

\PassOptionsToPackage{warn}{textcomp}
\usepackage[utf8]{inputenc}
\ifdefined\DeclareUnicodeCharacter
% support both utf8 and utf8x syntaxes
  \ifdefined\DeclareUnicodeCharacterAsOptional
    \def\sphinxDUC#1{\DeclareUnicodeCharacter{"#1}}
  \else
    \let\sphinxDUC\DeclareUnicodeCharacter
  \fi
  \sphinxDUC{00A0}{\nobreakspace}
  \sphinxDUC{2500}{\sphinxunichar{2500}}
  \sphinxDUC{2502}{\sphinxunichar{2502}}
  \sphinxDUC{2514}{\sphinxunichar{2514}}
  \sphinxDUC{251C}{\sphinxunichar{251C}}
  \sphinxDUC{2572}{\textbackslash}
\fi
\usepackage{cmap}
\usepackage[T1]{fontenc}
\usepackage{amsmath,amssymb,amstext}
\usepackage{babel}



\usepackage{tgtermes}
\usepackage{tgheros}
\renewcommand{\ttdefault}{txtt}



\usepackage[Bjarne]{fncychap}
\usepackage{sphinx}

\fvset{fontsize=auto}
\usepackage{geometry}


% Include hyperref last.
\usepackage{hyperref}
% Fix anchor placement for figures with captions.
\usepackage{hypcap}% it must be loaded after hyperref.
% Set up styles of URL: it should be placed after hyperref.
\urlstyle{same}


\usepackage{sphinxmessages}
\setcounter{tocdepth}{1}



\title{Forecasting framework}
\date{Aug 19, 2024}
\release{1}
\author{Gianluca Ferro}
\newcommand{\sphinxlogo}{\vbox{}}
\renewcommand{\releasename}{Release}
\makeindex
\begin{document}

\ifdefined\shorthandoff
  \ifnum\catcode`\=\string=\active\shorthandoff{=}\fi
  \ifnum\catcode`\"=\active\shorthandoff{"}\fi
\fi

\pagestyle{empty}
\sphinxmaketitle
\pagestyle{plain}
\sphinxtableofcontents
\pagestyle{normal}
\phantomsection\label{\detokenize{index::doc}}



\chapter{Introduction}
\label{\detokenize{index:introduction}}
\sphinxAtStartPar
This framework is designed to provide the main blocks for implementing and using many types of machine learning models for time series forecasting, including
statistical models, neural networks, and Extreme Gradient Boosting (XGB) models.
Other models can also be integrated, by incorporating the corresponding tools into each respective block of the framework.
The data preprocessing block makes possible to train and test the models on datasets with varying structures and formats, allowing a robust
support for handling NaN values and outliers. The framework comprises a main file that orchestrates the various implementation phases of the models, with initial settings provided as
command\sphinxhyphen{}line arguments using a parser (whose parameters are presented in the Appendix).
The code supports four distinct modes of operation: training, testing, combined training and testing, and fine tuning.
Various configurations of the framework, using different terminal arguments, are present in the JSON files (\sphinxtitleref{launch.json} for debug and \sphinxtitleref{tasks.json} for
code usage); however, using consistent command line arguments, it is possible to create custom configurations
by passing parameters directly through the terminal.


\chapter{Framework Architecture}
\label{\detokenize{index:framework-architecture}}
\sphinxAtStartPar
The main blocks of the framework are data loading, data preprocessing, training, testing, and performance measurement.
Each block makes use of classes and functions from a corresponding file located in the \sphinxtitleref{classes} folder,
in order to implement the time series forecasting models features.
For time series analysis and functions for loading/saving data there are two respective files in the \sphinxtitleref{utils} folder.
Below is the structure of the framework.

\sphinxstepscope


\section{Main Code}
\label{\detokenize{docs/main_code:module-main_code}}\label{\detokenize{docs/main_code:main-code}}\label{\detokenize{docs/main_code::doc}}\index{module@\spxentry{module}!main\_code@\spxentry{main\_code}}\index{main\_code@\spxentry{main\_code}!module@\spxentry{module}}\index{main() (in module main\_code)@\spxentry{main()}\spxextra{in module main\_code}}

\begin{fulllineitems}
\phantomsection\label{\detokenize{docs/main_code:main_code.main}}
\pysigstartsignatures
\pysiglinewithargsret{\sphinxcode{\sphinxupquote{main\_code.}}\sphinxbfcode{\sphinxupquote{main}}}{}{}
\pysigstopsignatures
\sphinxAtStartPar
Main function to execute time series forecasting tasks based on user\sphinxhyphen{}specified arguments.
This function handles the entire workflow from data loading, preprocessing, model training, testing,
and evaluation, based on the configuration provided via command\sphinxhyphen{}line arguments.

\end{fulllineitems}


\sphinxAtStartPar
The main code is divided into the following sections: ciaoooooooooooo
\begin{itemize}
\item {} 
\sphinxAtStartPar
\sphinxstylestrong{Data loading}

\end{itemize}

\sphinxAtStartPar
Loads data from the specified dataset path using the DataLoader class.
\begin{itemize}
\item {} 
\sphinxAtStartPar
\sphinxstylestrong{Preprocessing and dataset split}

\end{itemize}

\sphinxAtStartPar
Handling data preprocessing and dataset splitting based on the specified model type and runtime configurations.
For test\sphinxhyphen{}only run mode, only the test set will be created.
\begin{itemize}
\item {} 
\sphinxAtStartPar
\sphinxstylestrong{Optional time series analysis}

\end{itemize}

\sphinxAtStartPar
Performs additional time series analysis if specified with the parser argument \sphinxtitleref{\textendash{}ts\_analysis}.
Functions used in this section include plotting the ACF and PACF diagrams to have an early estimate of the AR and MA parameters,
as well as statistic tests for stationarity. For seasonal models, a seasonal\sphinxhyphen{}trend decomposition of the series can be performed.
\begin{itemize}
\item {} 
\sphinxAtStartPar
\sphinxstylestrong{Model loading for test or fine\sphinxhyphen{}tuning}

\end{itemize}

\sphinxAtStartPar
Depending on the run mode, loads a pre\sphinxhyphen{}trained model to perform fine\sphinxhyphen{}tuning or to handle the test\sphinxhyphen{}only mode.
For statistical models, some operations on the index of the training and test set are required, to ensure that the endogenous
and exogenous variables passed to the model during training have indexes that are contiguous to those of the pre\sphinxhyphen{}trained model.
\begin{itemize}
\item {} 
\sphinxAtStartPar
\sphinxstylestrong{Model training}

\end{itemize}

\sphinxAtStartPar
The training of the model is done using the corresponding method of the \sphinxtitleref{ModelTraining} class.
A buffer containing the last \sphinxtitleref{buffer\_size} elements of the training set is created, which can be used for
further test or to create naive benchmark models. After the training phase data (including the model itself and validation metrics)
are saved in the model’s directory.
\begin{itemize}
\item {} 
\sphinxAtStartPar
\sphinxstylestrong{Model testing}

\end{itemize}

\sphinxAtStartPar
The class ModelTest is used in order to generate the predictions of the trained model.
\begin{itemize}
\item {} 
\sphinxAtStartPar
\sphinxstylestrong{Plot of predictions versus real data}

\item {} 
\sphinxAtStartPar
\sphinxstylestrong{Performance measurement and saving}

\end{itemize}

\sphinxAtStartPar
After computing the metrics of the model, the performance is stored into a text file.

\sphinxAtStartPar
The \sphinxtitleref{main} function orchestrates the loading, preprocessing, training/testing of models, optionally fine\sphinxhyphen{}tuning, and plotting of predictions.

\sphinxstepscope


\section{Data Loader Module}
\label{\detokenize{docs/data_loader:data-loader-module}}\label{\detokenize{docs/data_loader::doc}}\index{module@\spxentry{module}!data\_loader@\spxentry{data\_loader}}\index{data\_loader@\spxentry{data\_loader}!module@\spxentry{module}}\index{DataLoader (class in data\_loader)@\spxentry{DataLoader}\spxextra{class in data\_loader}}\phantomsection\label{\detokenize{docs/data_loader:module-data_loader}}

\begin{fulllineitems}
\phantomsection\label{\detokenize{docs/data_loader:data_loader.DataLoader}}
\pysigstartsignatures
\pysiglinewithargsret{\sphinxbfcode{\sphinxupquote{class\DUrole{w}{ }}}\sphinxcode{\sphinxupquote{data\_loader.}}\sphinxbfcode{\sphinxupquote{DataLoader}}}{\sphinxparam{\DUrole{n}{file\_path}}\sphinxparamcomma \sphinxparam{\DUrole{n}{date\_format}}\sphinxparamcomma \sphinxparam{\DUrole{n}{model\_type}}\sphinxparamcomma \sphinxparam{\DUrole{n}{target\_column}}\sphinxparamcomma \sphinxparam{\DUrole{n}{time\_column\_index}\DUrole{o}{=}\DUrole{default_value}{0}}\sphinxparamcomma \sphinxparam{\DUrole{n}{date\_list}\DUrole{o}{=}\DUrole{default_value}{None}}\sphinxparamcomma \sphinxparam{\DUrole{n}{exog}\DUrole{o}{=}\DUrole{default_value}{None}}}{}
\pysigstopsignatures
\sphinxAtStartPar
Bases: \sphinxcode{\sphinxupquote{object}}

\sphinxAtStartPar
Class for loading datasets from various file formats and preparing them for machine learning models.
\begin{quote}\begin{description}
\sphinxlineitem{Parameters}\begin{itemize}
\item {} 
\sphinxAtStartPar
\sphinxstyleliteralstrong{\sphinxupquote{file\_path}} \textendash{} Path to the dataset file.

\item {} 
\sphinxAtStartPar
\sphinxstyleliteralstrong{\sphinxupquote{date\_format}} \textendash{} Format of the date in the dataset file, e.g., ‘\%Y\sphinxhyphen{}\%m\sphinxhyphen{}\%d’.

\item {} 
\sphinxAtStartPar
\sphinxstyleliteralstrong{\sphinxupquote{model\_type}} \textendash{} Type of the machine learning model. Supported models are ‘LSTM’, ‘XGB’, ‘ARIMA’, ‘SARIMA’, ‘SARIMAX’.

\item {} 
\sphinxAtStartPar
\sphinxstyleliteralstrong{\sphinxupquote{target\_column}} \textendash{} Name of the target column in the dataset.

\item {} 
\sphinxAtStartPar
\sphinxstyleliteralstrong{\sphinxupquote{time\_column\_index}} \textendash{} Index of the time column in the dataset (default is 0).

\item {} 
\sphinxAtStartPar
\sphinxstyleliteralstrong{\sphinxupquote{date\_list}} \textendash{} List of specific dates to be filtered (default is None).

\item {} 
\sphinxAtStartPar
\sphinxstyleliteralstrong{\sphinxupquote{exog}} \textendash{} Name or list of exogenous variables (default is None).

\end{itemize}

\end{description}\end{quote}
\index{load\_data() (data\_loader.DataLoader method)@\spxentry{load\_data()}\spxextra{data\_loader.DataLoader method}}

\begin{fulllineitems}
\phantomsection\label{\detokenize{docs/data_loader:data_loader.DataLoader.load_data}}
\pysigstartsignatures
\pysiglinewithargsret{\sphinxbfcode{\sphinxupquote{load\_data}}}{}{}
\pysigstopsignatures
\sphinxAtStartPar
Loads data from a file, processes it according to the specified settings,
and prepares it for machine learning models. This includes formatting date columns,
filtering specific dates, and adjusting data structure based on the model type.
\begin{quote}\begin{description}
\sphinxlineitem{Returns}
\sphinxAtStartPar
\begin{itemize}
\item {} 
\sphinxAtStartPar
A tuple containing the dataframe and the indices of the dates if provided in \sphinxtitleref{date\_list}.

\end{itemize}


\end{description}\end{quote}

\end{fulllineitems}


\end{fulllineitems}


\sphinxAtStartPar
Class for loading data from various file formats. If the parameter \textendash{}time\_column\_index is not specified,
the code expects that the time column is the first of the dataset.
For statistical models, the end date of the training set must be equal to the start date of the validation set;
the same holds for validation and test set.

\sphinxstepscope


\section{Data Preprocessing}
\label{\detokenize{docs/data_preprocessing:data-preprocessing}}\label{\detokenize{docs/data_preprocessing::doc}}\index{module@\spxentry{module}!data\_preprocessing@\spxentry{data\_preprocessing}}\index{data\_preprocessing@\spxentry{data\_preprocessing}!module@\spxentry{module}}\index{DataPreprocessor (class in data\_preprocessing)@\spxentry{DataPreprocessor}\spxextra{class in data\_preprocessing}}\phantomsection\label{\detokenize{docs/data_preprocessing:module-data_preprocessing}}

\begin{fulllineitems}
\phantomsection\label{\detokenize{docs/data_preprocessing:data_preprocessing.DataPreprocessor}}
\pysigstartsignatures
\pysiglinewithargsret{\sphinxbfcode{\sphinxupquote{class\DUrole{w}{ }}}\sphinxcode{\sphinxupquote{data\_preprocessing.}}\sphinxbfcode{\sphinxupquote{DataPreprocessor}}}{\sphinxparam{\DUrole{n}{file\_ext}}\sphinxparamcomma \sphinxparam{\DUrole{n}{run\_mode}}\sphinxparamcomma \sphinxparam{\DUrole{n}{model\_type}}\sphinxparamcomma \sphinxparam{\DUrole{n}{df}\DUrole{p}{:}\DUrole{w}{ }\DUrole{n}{DataFrame}}\sphinxparamcomma \sphinxparam{\DUrole{n}{target\_column}\DUrole{p}{:}\DUrole{w}{ }\DUrole{n}{str}}\sphinxparamcomma \sphinxparam{\DUrole{n}{dates}\DUrole{o}{=}\DUrole{default_value}{None}}\sphinxparamcomma \sphinxparam{\DUrole{n}{scaling}\DUrole{o}{=}\DUrole{default_value}{False}}\sphinxparamcomma \sphinxparam{\DUrole{n}{validation}\DUrole{o}{=}\DUrole{default_value}{None}}\sphinxparamcomma \sphinxparam{\DUrole{n}{train\_size}\DUrole{o}{=}\DUrole{default_value}{0.7}}\sphinxparamcomma \sphinxparam{\DUrole{n}{val\_size}\DUrole{o}{=}\DUrole{default_value}{0.2}}\sphinxparamcomma \sphinxparam{\DUrole{n}{test\_size}\DUrole{o}{=}\DUrole{default_value}{0.1}}\sphinxparamcomma \sphinxparam{\DUrole{n}{folder\_path}\DUrole{o}{=}\DUrole{default_value}{None}}\sphinxparamcomma \sphinxparam{\DUrole{n}{model\_path}\DUrole{o}{=}\DUrole{default_value}{None}}\sphinxparamcomma \sphinxparam{\DUrole{n}{verbose}\DUrole{o}{=}\DUrole{default_value}{False}}}{}
\pysigstopsignatures
\sphinxAtStartPar
Bases: \sphinxcode{\sphinxupquote{object}}

\sphinxAtStartPar
A class to handle operations of preprocessing, including tasks such as managing NaN values,
removing non\sphinxhyphen{}numeric columns, splitting datasets, managing outliers, and scaling data.
\begin{quote}\begin{description}
\sphinxlineitem{Parameters}\begin{itemize}
\item {} 
\sphinxAtStartPar
\sphinxstyleliteralstrong{\sphinxupquote{file\_ext}} \textendash{} File extension for saving datasets.

\item {} 
\sphinxAtStartPar
\sphinxstyleliteralstrong{\sphinxupquote{run\_mode}} \textendash{} Mode of operation (‘train’, ‘test’, ‘train\_test’, ‘fine\_tuning’).

\item {} 
\sphinxAtStartPar
\sphinxstyleliteralstrong{\sphinxupquote{model\_type}} \textendash{} Type of machine learning model to prepare data for.

\item {} 
\sphinxAtStartPar
\sphinxstyleliteralstrong{\sphinxupquote{df}} \textendash{} DataFrame containing the data.

\item {} 
\sphinxAtStartPar
\sphinxstyleliteralstrong{\sphinxupquote{target\_column}} \textendash{} Name of the target column in the DataFrame.

\item {} 
\sphinxAtStartPar
\sphinxstyleliteralstrong{\sphinxupquote{dates}} \textendash{} Indexes of dates given by command line with \textendash{}date\_list.

\item {} 
\sphinxAtStartPar
\sphinxstyleliteralstrong{\sphinxupquote{scaling}} \textendash{} Boolean flag to determine if scaling should be applied.

\item {} 
\sphinxAtStartPar
\sphinxstyleliteralstrong{\sphinxupquote{validation}} \textendash{} Boolean flag to determine if a validation set should be created.

\item {} 
\sphinxAtStartPar
\sphinxstyleliteralstrong{\sphinxupquote{train\_size}} \textendash{} Proportion of data to be used for training.

\item {} 
\sphinxAtStartPar
\sphinxstyleliteralstrong{\sphinxupquote{val\_size}} \textendash{} Proportion of data to be used for validation.

\item {} 
\sphinxAtStartPar
\sphinxstyleliteralstrong{\sphinxupquote{test\_size}} \textendash{} Proportion of data to be used for testing.

\item {} 
\sphinxAtStartPar
\sphinxstyleliteralstrong{\sphinxupquote{folder\_path}} \textendash{} Path to folder for saving data.

\item {} 
\sphinxAtStartPar
\sphinxstyleliteralstrong{\sphinxupquote{model\_path}} \textendash{} Path to model file for loading or saving the model.

\item {} 
\sphinxAtStartPar
\sphinxstyleliteralstrong{\sphinxupquote{verbose}} \textendash{} Boolean flag for verbose output.

\end{itemize}

\end{description}\end{quote}
\index{conditional\_print() (data\_preprocessing.DataPreprocessor method)@\spxentry{conditional\_print()}\spxextra{data\_preprocessing.DataPreprocessor method}}

\begin{fulllineitems}
\phantomsection\label{\detokenize{docs/data_preprocessing:data_preprocessing.DataPreprocessor.conditional_print}}
\pysigstartsignatures
\pysiglinewithargsret{\sphinxbfcode{\sphinxupquote{conditional\_print}}}{\sphinxparam{\DUrole{o}{*}\DUrole{n}{args}}\sphinxparamcomma \sphinxparam{\DUrole{o}{**}\DUrole{n}{kwargs}}}{}
\pysigstopsignatures
\sphinxAtStartPar
Print messages conditionally based on the verbose attribute.
\begin{quote}\begin{description}
\sphinxlineitem{Parameters}\begin{itemize}
\item {} 
\sphinxAtStartPar
\sphinxstyleliteralstrong{\sphinxupquote{args}} \textendash{} Non\sphinxhyphen{}keyword arguments to be printed

\item {} 
\sphinxAtStartPar
\sphinxstyleliteralstrong{\sphinxupquote{kwargs}} \textendash{} Keyword arguments to be printed

\end{itemize}

\end{description}\end{quote}

\end{fulllineitems}

\index{create\_time\_features() (data\_preprocessing.DataPreprocessor method)@\spxentry{create\_time\_features()}\spxextra{data\_preprocessing.DataPreprocessor method}}

\begin{fulllineitems}
\phantomsection\label{\detokenize{docs/data_preprocessing:data_preprocessing.DataPreprocessor.create_time_features}}
\pysigstartsignatures
\pysiglinewithargsret{\sphinxbfcode{\sphinxupquote{create\_time\_features}}}{\sphinxparam{\DUrole{n}{df}}\sphinxparamcomma \sphinxparam{\DUrole{n}{label}\DUrole{o}{=}\DUrole{default_value}{None}}\sphinxparamcomma \sphinxparam{\DUrole{n}{seasonal\_model}\DUrole{o}{=}\DUrole{default_value}{None}}\sphinxparamcomma \sphinxparam{\DUrole{n}{lags}\DUrole{o}{=}\DUrole{default_value}{{[}1, 2, 3, 24{]}}}\sphinxparamcomma \sphinxparam{\DUrole{n}{rolling\_window}\DUrole{o}{=}\DUrole{default_value}{24}}}{}
\pysigstopsignatures
\sphinxAtStartPar
Create time\sphinxhyphen{}based features for a DataFrame, optionally including Fourier features and rolling window statistics.
\begin{quote}\begin{description}
\sphinxlineitem{Parameters}\begin{itemize}
\item {} 
\sphinxAtStartPar
\sphinxstyleliteralstrong{\sphinxupquote{df}} \textendash{} DataFrame to modify with time\sphinxhyphen{}based features.

\item {} 
\sphinxAtStartPar
\sphinxstyleliteralstrong{\sphinxupquote{label}} \textendash{} Label column name for generating features.

\item {} 
\sphinxAtStartPar
\sphinxstyleliteralstrong{\sphinxupquote{seasonal\_model}} \textendash{} Boolean indicating whether to add Fourier features for seasonal models.

\item {} 
\sphinxAtStartPar
\sphinxstyleliteralstrong{\sphinxupquote{lags}} \textendash{} List of integers representing lag periods to generate features for.

\item {} 
\sphinxAtStartPar
\sphinxstyleliteralstrong{\sphinxupquote{rolling\_window}} \textendash{} Window size for generating rolling mean and standard deviation.

\end{itemize}

\sphinxlineitem{Returns}
\sphinxAtStartPar
Modified DataFrame with new features, optionally including target column labels.

\end{description}\end{quote}

\end{fulllineitems}

\index{data\_windowing() (data\_preprocessing.DataPreprocessor method)@\spxentry{data\_windowing()}\spxextra{data\_preprocessing.DataPreprocessor method}}

\begin{fulllineitems}
\phantomsection\label{\detokenize{docs/data_preprocessing:data_preprocessing.DataPreprocessor.data_windowing}}
\pysigstartsignatures
\pysiglinewithargsret{\sphinxbfcode{\sphinxupquote{data\_windowing}}}{\sphinxparam{\DUrole{n}{train}}\sphinxparamcomma \sphinxparam{\DUrole{n}{valid}}\sphinxparamcomma \sphinxparam{\DUrole{n}{test}}\sphinxparamcomma \sphinxparam{\DUrole{n}{input\_len}}\sphinxparamcomma \sphinxparam{\DUrole{n}{output\_len}}\sphinxparamcomma \sphinxparam{\DUrole{n}{seasonal\_model}\DUrole{o}{=}\DUrole{default_value}{False}}\sphinxparamcomma \sphinxparam{\DUrole{n}{set\_fourier}\DUrole{o}{=}\DUrole{default_value}{False}}}{}
\pysigstopsignatures
\sphinxAtStartPar
Creates data windows suitable for input into deep learning models, optionally incorporating Fourier features for seasonality.
\begin{quote}\begin{description}
\sphinxlineitem{Parameters}\begin{itemize}
\item {} 
\sphinxAtStartPar
\sphinxstyleliteralstrong{\sphinxupquote{train}} \textendash{} Training dataset.

\item {} 
\sphinxAtStartPar
\sphinxstyleliteralstrong{\sphinxupquote{valid}} \textendash{} Validation dataset.

\item {} 
\sphinxAtStartPar
\sphinxstyleliteralstrong{\sphinxupquote{test}} \textendash{} Test dataset.

\item {} 
\sphinxAtStartPar
\sphinxstyleliteralstrong{\sphinxupquote{input\_len}} \textendash{} Length of the input data window.

\item {} 
\sphinxAtStartPar
\sphinxstyleliteralstrong{\sphinxupquote{output\_len}} \textendash{} Length of the output data window.

\item {} 
\sphinxAtStartPar
\sphinxstyleliteralstrong{\sphinxupquote{seasonal\_model}} \textendash{} Flag to include Fourier features for seasonal predictions.

\item {} 
\sphinxAtStartPar
\sphinxstyleliteralstrong{\sphinxupquote{set\_fourier}} \textendash{} Flag to set Fourier transformation features.

\end{itemize}

\sphinxlineitem{Returns}
\sphinxAtStartPar
Arrays of input and output data windows for training, validation, and testing.

\end{description}\end{quote}

\end{fulllineitems}

\index{detect\_nan\_hole() (data\_preprocessing.DataPreprocessor method)@\spxentry{detect\_nan\_hole()}\spxextra{data\_preprocessing.DataPreprocessor method}}

\begin{fulllineitems}
\phantomsection\label{\detokenize{docs/data_preprocessing:data_preprocessing.DataPreprocessor.detect_nan_hole}}
\pysigstartsignatures
\pysiglinewithargsret{\sphinxbfcode{\sphinxupquote{detect\_nan\_hole}}}{\sphinxparam{\DUrole{n}{df}}}{}
\pysigstopsignatures
\sphinxAtStartPar
Detects the largest contiguous NaN hole in the target column.
\begin{quote}\begin{description}
\sphinxlineitem{Parameters}
\sphinxAtStartPar
\sphinxstyleliteralstrong{\sphinxupquote{df}} \textendash{} DataFrame in which to find the NaN hole

\sphinxlineitem{Returns}
\sphinxAtStartPar
A dictionary with the start and end indices of the largest NaN hole in the target column

\end{description}\end{quote}

\end{fulllineitems}

\index{manage\_nan() (data\_preprocessing.DataPreprocessor method)@\spxentry{manage\_nan()}\spxextra{data\_preprocessing.DataPreprocessor method}}

\begin{fulllineitems}
\phantomsection\label{\detokenize{docs/data_preprocessing:data_preprocessing.DataPreprocessor.manage_nan}}
\pysigstartsignatures
\pysiglinewithargsret{\sphinxbfcode{\sphinxupquote{manage\_nan}}}{\sphinxparam{\DUrole{n}{df}}\sphinxparamcomma \sphinxparam{\DUrole{n}{max\_nan\_percentage}\DUrole{o}{=}\DUrole{default_value}{50}}\sphinxparamcomma \sphinxparam{\DUrole{n}{min\_nan\_percentage}\DUrole{o}{=}\DUrole{default_value}{10}}\sphinxparamcomma \sphinxparam{\DUrole{n}{percent\_threshold}\DUrole{o}{=}\DUrole{default_value}{40}}}{}
\pysigstopsignatures
\sphinxAtStartPar
Manage NaN values in the dataset based on defined percentage thresholds and interpolation strategies.
\begin{quote}\begin{description}
\sphinxlineitem{Parameters}\begin{itemize}
\item {} 
\sphinxAtStartPar
\sphinxstyleliteralstrong{\sphinxupquote{df}} \textendash{} Dataframe to analyze

\item {} 
\sphinxAtStartPar
\sphinxstyleliteralstrong{\sphinxupquote{max\_nan\_percentage}} \textendash{} Maximum allowed percentage of NaN values for a column to be interpolated or kept

\item {} 
\sphinxAtStartPar
\sphinxstyleliteralstrong{\sphinxupquote{min\_nan\_percentage}} \textendash{} Minimum percentage of NaN values for which linear interpolation is applied

\item {} 
\sphinxAtStartPar
\sphinxstyleliteralstrong{\sphinxupquote{percent\_threshold}} \textendash{} Threshold percentage of NaNs in the target column to decide between interpolation and splitting the dataset

\end{itemize}

\sphinxlineitem{Returns}
\sphinxAtStartPar
A tuple (df, exit), where df is the DataFrame after NaN management, and exit is a boolean flag indicating if the dataset needs to be split

\end{description}\end{quote}

\end{fulllineitems}

\index{preprocess\_data() (data\_preprocessing.DataPreprocessor method)@\spxentry{preprocess\_data()}\spxextra{data\_preprocessing.DataPreprocessor method}}

\begin{fulllineitems}
\phantomsection\label{\detokenize{docs/data_preprocessing:data_preprocessing.DataPreprocessor.preprocess_data}}
\pysigstartsignatures
\pysiglinewithargsret{\sphinxbfcode{\sphinxupquote{preprocess\_data}}}{}{}
\pysigstopsignatures
\sphinxAtStartPar
Main method to preprocess the dataset according to specified configurations.
\begin{quote}\begin{description}
\sphinxlineitem{Returns}
\sphinxAtStartPar
Depending on the mode, returns the splitted dataframe and an exit flag.

\end{description}\end{quote}

\end{fulllineitems}

\index{print\_stats() (data\_preprocessing.DataPreprocessor method)@\spxentry{print\_stats()}\spxextra{data\_preprocessing.DataPreprocessor method}}

\begin{fulllineitems}
\phantomsection\label{\detokenize{docs/data_preprocessing:data_preprocessing.DataPreprocessor.print_stats}}
\pysigstartsignatures
\pysiglinewithargsret{\sphinxbfcode{\sphinxupquote{print\_stats}}}{\sphinxparam{\DUrole{n}{train}}}{}
\pysigstopsignatures
\sphinxAtStartPar
Print statistics for the selected feature in the training dataset.
\begin{quote}\begin{description}
\sphinxlineitem{Parameters}
\sphinxAtStartPar
\sphinxstyleliteralstrong{\sphinxupquote{train}} \textendash{} DataFrame containing the training data

\end{description}\end{quote}

\end{fulllineitems}

\index{replace\_outliers() (data\_preprocessing.DataPreprocessor method)@\spxentry{replace\_outliers()}\spxextra{data\_preprocessing.DataPreprocessor method}}

\begin{fulllineitems}
\phantomsection\label{\detokenize{docs/data_preprocessing:data_preprocessing.DataPreprocessor.replace_outliers}}
\pysigstartsignatures
\pysiglinewithargsret{\sphinxbfcode{\sphinxupquote{replace\_outliers}}}{\sphinxparam{\DUrole{n}{df}}}{}
\pysigstopsignatures
\sphinxAtStartPar
Replaces outliers in the dataset based on the Interquartile Range (IQR)
method. Instead of analyzing the entire dataset at once, this method focuses on a window of data points at a time.
The window moves through the data series step by step. For each step, it includes the next data point
in the sequence while dropping the oldest one, thus maintaining a constant
window size. For each position of the window, the function calculates the
first (Q1) and third (Q3) quartiles of the data within the window. These
quartiles are used to determine the Interquartile Range (IQR), from which
lower and upper bounds for outliers are derived.
\begin{quote}\begin{description}
\sphinxlineitem{Parameters}
\sphinxAtStartPar
\sphinxstyleliteralstrong{\sphinxupquote{df}} \textendash{} DataFrame from which to remove and replace outliers

\sphinxlineitem{Returns}
\sphinxAtStartPar
DataFrame with outliers replaced

\end{description}\end{quote}

\end{fulllineitems}

\index{split\_data() (data\_preprocessing.DataPreprocessor method)@\spxentry{split\_data()}\spxextra{data\_preprocessing.DataPreprocessor method}}

\begin{fulllineitems}
\phantomsection\label{\detokenize{docs/data_preprocessing:data_preprocessing.DataPreprocessor.split_data}}
\pysigstartsignatures
\pysiglinewithargsret{\sphinxbfcode{\sphinxupquote{split\_data}}}{\sphinxparam{\DUrole{n}{df}}}{}
\pysigstopsignatures
\sphinxAtStartPar
Split the dataset into training, validation, and test sets.
If a list with dates is given, each set is created within the respective dates, otherwise the sets are created following
the given percentage sizes.
\begin{quote}\begin{description}
\sphinxlineitem{Parameters}
\sphinxAtStartPar
\sphinxstyleliteralstrong{\sphinxupquote{df}} \textendash{} DataFrame to split

\sphinxlineitem{Returns}
\sphinxAtStartPar
Tuple of DataFrames for training, testing, and validation

\end{description}\end{quote}

\end{fulllineitems}

\index{split\_file\_at\_nanhole() (data\_preprocessing.DataPreprocessor method)@\spxentry{split\_file\_at\_nanhole()}\spxextra{data\_preprocessing.DataPreprocessor method}}

\begin{fulllineitems}
\phantomsection\label{\detokenize{docs/data_preprocessing:data_preprocessing.DataPreprocessor.split_file_at_nanhole}}
\pysigstartsignatures
\pysiglinewithargsret{\sphinxbfcode{\sphinxupquote{split\_file\_at\_nanhole}}}{\sphinxparam{\DUrole{n}{nan\_hole}}}{}
\pysigstopsignatures
\sphinxAtStartPar
Splits the dataset at a significant NaN hole into two separate files.
\begin{quote}\begin{description}
\sphinxlineitem{Parameters}
\sphinxAtStartPar
\sphinxstyleliteralstrong{\sphinxupquote{nan\_hole}} \textendash{} Dictionary containing start and end indices of the NaN hole in the target column

\end{description}\end{quote}

\end{fulllineitems}


\end{fulllineitems}


\sphinxAtStartPar
Class for preprocessing data for machine learning models.

\sphinxstepscope


\section{Training Module}
\label{\detokenize{docs/training_module:module-training_module}}\label{\detokenize{docs/training_module:training-module}}\label{\detokenize{docs/training_module::doc}}\index{module@\spxentry{module}!training\_module@\spxentry{training\_module}}\index{training\_module@\spxentry{training\_module}!module@\spxentry{module}}\index{ModelTraining (class in training\_module)@\spxentry{ModelTraining}\spxextra{class in training\_module}}

\begin{fulllineitems}
\phantomsection\label{\detokenize{docs/training_module:training_module.ModelTraining}}
\pysigstartsignatures
\pysiglinewithargsret{\sphinxbfcode{\sphinxupquote{class\DUrole{w}{ }}}\sphinxcode{\sphinxupquote{training\_module.}}\sphinxbfcode{\sphinxupquote{ModelTraining}}}{\sphinxparam{\DUrole{n}{model\_type}\DUrole{p}{:}\DUrole{w}{ }\DUrole{n}{str}}\sphinxparamcomma \sphinxparam{\DUrole{n}{train}}\sphinxparamcomma \sphinxparam{\DUrole{n}{valid}\DUrole{o}{=}\DUrole{default_value}{None}}\sphinxparamcomma \sphinxparam{\DUrole{n}{valid\_steps}\DUrole{o}{=}\DUrole{default_value}{None}}\sphinxparamcomma \sphinxparam{\DUrole{n}{target\_column}\DUrole{o}{=}\DUrole{default_value}{None}}\sphinxparamcomma \sphinxparam{\DUrole{n}{verbose}\DUrole{o}{=}\DUrole{default_value}{False}}}{}
\pysigstopsignatures
\sphinxAtStartPar
Bases: \sphinxcode{\sphinxupquote{object}}

\sphinxAtStartPar
Class for training various types of machine learning models based on the \textendash{}model\_type argument.
\begin{quote}\begin{description}
\sphinxlineitem{Parameters}\begin{itemize}
\item {} 
\sphinxAtStartPar
\sphinxstyleliteralstrong{\sphinxupquote{model\_type}} \textendash{} Specifies the type of model to train (e.g., ‘ARIMA’, ‘SARIMAX’, ‘LSTM’, ‘XGB’).

\item {} 
\sphinxAtStartPar
\sphinxstyleliteralstrong{\sphinxupquote{train}} \textendash{} Training dataset.

\item {} 
\sphinxAtStartPar
\sphinxstyleliteralstrong{\sphinxupquote{valid}} \textendash{} Optional validation dataset for model evaluation.

\item {} 
\sphinxAtStartPar
\sphinxstyleliteralstrong{\sphinxupquote{target\_column}} \textendash{} The name of the target variable in the dataset.

\item {} 
\sphinxAtStartPar
\sphinxstyleliteralstrong{\sphinxupquote{verbose}} \textendash{} If True, enables verbose output during model training.

\end{itemize}

\end{description}\end{quote}
\index{train\_ARIMA\_model() (training\_module.ModelTraining method)@\spxentry{train\_ARIMA\_model()}\spxextra{training\_module.ModelTraining method}}

\begin{fulllineitems}
\phantomsection\label{\detokenize{docs/training_module:training_module.ModelTraining.train_ARIMA_model}}
\pysigstartsignatures
\pysiglinewithargsret{\sphinxbfcode{\sphinxupquote{train\_ARIMA\_model}}}{}{}
\pysigstopsignatures
\sphinxAtStartPar
Trains an ARIMA model using the training dataset.
\begin{quote}\begin{description}
\sphinxlineitem{Returns}
\sphinxAtStartPar
A tuple containing the trained model, validation metrics and the index of last training/validation timestep.

\end{description}\end{quote}

\end{fulllineitems}

\index{train\_LSTM\_model() (training\_module.ModelTraining method)@\spxentry{train\_LSTM\_model()}\spxextra{training\_module.ModelTraining method}}

\begin{fulllineitems}
\phantomsection\label{\detokenize{docs/training_module:training_module.ModelTraining.train_LSTM_model}}
\pysigstartsignatures
\pysiglinewithargsret{\sphinxbfcode{\sphinxupquote{train\_LSTM\_model}}}{\sphinxparam{\DUrole{n}{X\_train}}\sphinxparamcomma \sphinxparam{\DUrole{n}{y\_train}}\sphinxparamcomma \sphinxparam{\DUrole{n}{X\_valid}}\sphinxparamcomma \sphinxparam{\DUrole{n}{y\_valid}}\sphinxparamcomma \sphinxparam{\DUrole{n}{output\_len}}}{}
\pysigstopsignatures
\sphinxAtStartPar
Trains an LSTM model using the training and validation datasets.
\begin{quote}\begin{description}
\sphinxlineitem{Parameters}\begin{itemize}
\item {} 
\sphinxAtStartPar
\sphinxstyleliteralstrong{\sphinxupquote{X\_train}} \textendash{} Input data for training.

\item {} 
\sphinxAtStartPar
\sphinxstyleliteralstrong{\sphinxupquote{y\_train}} \textendash{} Target variable for training.

\item {} 
\sphinxAtStartPar
\sphinxstyleliteralstrong{\sphinxupquote{X\_valid}} \textendash{} Input data for validation.

\item {} 
\sphinxAtStartPar
\sphinxstyleliteralstrong{\sphinxupquote{y\_valid}} \textendash{} Target variable for validation.

\item {} 
\sphinxAtStartPar
\sphinxstyleliteralstrong{\sphinxupquote{output\_len}} \textendash{} The length of the output sequence for the LSTM model.

\end{itemize}

\sphinxlineitem{Returns}
\sphinxAtStartPar
A tuple containing the trained LSTM model and validation metrics.

\end{description}\end{quote}

\end{fulllineitems}

\index{train\_SARIMAX\_model() (training\_module.ModelTraining method)@\spxentry{train\_SARIMAX\_model()}\spxextra{training\_module.ModelTraining method}}

\begin{fulllineitems}
\phantomsection\label{\detokenize{docs/training_module:training_module.ModelTraining.train_SARIMAX_model}}
\pysigstartsignatures
\pysiglinewithargsret{\sphinxbfcode{\sphinxupquote{train\_SARIMAX\_model}}}{\sphinxparam{\DUrole{n}{target\_train}}\sphinxparamcomma \sphinxparam{\DUrole{n}{exog\_train}}\sphinxparamcomma \sphinxparam{\DUrole{n}{exog\_valid}\DUrole{o}{=}\DUrole{default_value}{None}}\sphinxparamcomma \sphinxparam{\DUrole{n}{period}\DUrole{o}{=}\DUrole{default_value}{24}}\sphinxparamcomma \sphinxparam{\DUrole{n}{set\_Fourier}\DUrole{o}{=}\DUrole{default_value}{False}}}{}
\pysigstopsignatures
\sphinxAtStartPar
Trains a SARIMAX model using the training dataset and exogenous variables.
\begin{quote}\begin{description}
\sphinxlineitem{Parameters}\begin{itemize}
\item {} 
\sphinxAtStartPar
\sphinxstyleliteralstrong{\sphinxupquote{target\_train}} \textendash{} Training dataset containing the target variable.

\item {} 
\sphinxAtStartPar
\sphinxstyleliteralstrong{\sphinxupquote{exog\_train}} \textendash{} Training dataset containing the exogenous variables.

\item {} 
\sphinxAtStartPar
\sphinxstyleliteralstrong{\sphinxupquote{exog\_valid}} \textendash{} Optional validation dataset containing the exogenous variables for model evaluation.

\item {} 
\sphinxAtStartPar
\sphinxstyleliteralstrong{\sphinxupquote{period}} \textendash{} Seasonal period of the SARIMAX model.

\item {} 
\sphinxAtStartPar
\sphinxstyleliteralstrong{\sphinxupquote{set\_Fourier}} \textendash{} Boolean flag to determine if Fourier terms should be included.

\end{itemize}

\sphinxlineitem{Returns}
\sphinxAtStartPar
A tuple containing the trained model, validation metrics and the index of last training/validation timestep.

\end{description}\end{quote}

\end{fulllineitems}

\index{train\_XGB\_model() (training\_module.ModelTraining method)@\spxentry{train\_XGB\_model()}\spxextra{training\_module.ModelTraining method}}

\begin{fulllineitems}
\phantomsection\label{\detokenize{docs/training_module:training_module.ModelTraining.train_XGB_model}}
\pysigstartsignatures
\pysiglinewithargsret{\sphinxbfcode{\sphinxupquote{train\_XGB\_model}}}{\sphinxparam{\DUrole{n}{X\_train}}\sphinxparamcomma \sphinxparam{\DUrole{n}{y\_train}}\sphinxparamcomma \sphinxparam{\DUrole{n}{X\_valid}}\sphinxparamcomma \sphinxparam{\DUrole{n}{y\_valid}}}{}
\pysigstopsignatures
\sphinxAtStartPar
Trains an XGBoost model using the training and validation datasets.
\begin{quote}\begin{description}
\sphinxlineitem{Parameters}\begin{itemize}
\item {} 
\sphinxAtStartPar
\sphinxstyleliteralstrong{\sphinxupquote{X\_train}} \textendash{} Input data for training.

\item {} 
\sphinxAtStartPar
\sphinxstyleliteralstrong{\sphinxupquote{y\_train}} \textendash{} Target variable for training.

\item {} 
\sphinxAtStartPar
\sphinxstyleliteralstrong{\sphinxupquote{X\_valid}} \textendash{} Input data for validation.

\item {} 
\sphinxAtStartPar
\sphinxstyleliteralstrong{\sphinxupquote{y\_valid}} \textendash{} Target variable for validation.

\end{itemize}

\sphinxlineitem{Returns}
\sphinxAtStartPar
A tuple containing the trained XGBoost model and validation metrics.

\end{description}\end{quote}

\end{fulllineitems}


\end{fulllineitems}


\sphinxAtStartPar
Class for training time series forecasting models.

\sphinxstepscope


\section{Testing Module}
\label{\detokenize{docs/model_testing:module-model_testing}}\label{\detokenize{docs/model_testing:testing-module}}\label{\detokenize{docs/model_testing::doc}}\index{module@\spxentry{module}!model\_testing@\spxentry{model\_testing}}\index{model\_testing@\spxentry{model\_testing}!module@\spxentry{module}}\index{ModelTest (class in model\_testing)@\spxentry{ModelTest}\spxextra{class in model\_testing}}

\begin{fulllineitems}
\phantomsection\label{\detokenize{docs/model_testing:model_testing.ModelTest}}
\pysigstartsignatures
\pysiglinewithargsret{\sphinxbfcode{\sphinxupquote{class\DUrole{w}{ }}}\sphinxcode{\sphinxupquote{model\_testing.}}\sphinxbfcode{\sphinxupquote{ModelTest}}}{\sphinxparam{\DUrole{n}{model\_type}}\sphinxparamcomma \sphinxparam{\DUrole{n}{model}}\sphinxparamcomma \sphinxparam{\DUrole{n}{test}}\sphinxparamcomma \sphinxparam{\DUrole{n}{target\_column}}\sphinxparamcomma \sphinxparam{\DUrole{n}{forecast\_type}}}{}
\pysigstopsignatures
\sphinxAtStartPar
Bases: \sphinxcode{\sphinxupquote{object}}

\sphinxAtStartPar
A class for testing and visualizing the predictions of various types of forecasting models.
\begin{quote}\begin{description}
\sphinxlineitem{Parameters}\begin{itemize}
\item {} 
\sphinxAtStartPar
\sphinxstyleliteralstrong{\sphinxupquote{model\_type}} \textendash{} The type of model to test (‘ARIMA’, ‘SARIMAX’, etc.).

\item {} 
\sphinxAtStartPar
\sphinxstyleliteralstrong{\sphinxupquote{model}} \textendash{} The model object to be tested.

\item {} 
\sphinxAtStartPar
\sphinxstyleliteralstrong{\sphinxupquote{test}} \textendash{} The test set.

\item {} 
\sphinxAtStartPar
\sphinxstyleliteralstrong{\sphinxupquote{target\_column}} \textendash{} The target column in the dataset.

\item {} 
\sphinxAtStartPar
\sphinxstyleliteralstrong{\sphinxupquote{forecast\_type}} \textendash{} The type of forecasting to be performed (‘ol\sphinxhyphen{}one’, etc.).

\end{itemize}

\end{description}\end{quote}
\index{ARIMA\_plot\_pred() (model\_testing.ModelTest method)@\spxentry{ARIMA\_plot\_pred()}\spxextra{model\_testing.ModelTest method}}

\begin{fulllineitems}
\phantomsection\label{\detokenize{docs/model_testing:model_testing.ModelTest.ARIMA_plot_pred}}
\pysigstartsignatures
\pysiglinewithargsret{\sphinxbfcode{\sphinxupquote{ARIMA\_plot\_pred}}}{\sphinxparam{\DUrole{n}{best\_order}}\sphinxparamcomma \sphinxparam{\DUrole{n}{predictions}}\sphinxparamcomma \sphinxparam{\DUrole{n}{naive\_predictions}\DUrole{o}{=}\DUrole{default_value}{None}}}{}
\pysigstopsignatures
\sphinxAtStartPar
Plots the ARIMA model predictions against the test data and optionally against naive predictions.
\begin{quote}\begin{description}
\sphinxlineitem{Parameters}\begin{itemize}
\item {} 
\sphinxAtStartPar
\sphinxstyleliteralstrong{\sphinxupquote{best\_order}} \textendash{} The order of the ARIMA model used.

\item {} 
\sphinxAtStartPar
\sphinxstyleliteralstrong{\sphinxupquote{predictions}} \textendash{} The predictions made by the ARIMA model.

\item {} 
\sphinxAtStartPar
\sphinxstyleliteralstrong{\sphinxupquote{naive\_predictions}} \textendash{} Optional naive predictions for comparison.

\end{itemize}

\end{description}\end{quote}

\end{fulllineitems}

\index{LSTM\_plot\_pred() (model\_testing.ModelTest method)@\spxentry{LSTM\_plot\_pred()}\spxextra{model\_testing.ModelTest method}}

\begin{fulllineitems}
\phantomsection\label{\detokenize{docs/model_testing:model_testing.ModelTest.LSTM_plot_pred}}
\pysigstartsignatures
\pysiglinewithargsret{\sphinxbfcode{\sphinxupquote{LSTM\_plot\_pred}}}{\sphinxparam{\DUrole{n}{predictions}}\sphinxparamcomma \sphinxparam{\DUrole{n}{y\_test}}}{}
\pysigstopsignatures
\sphinxAtStartPar
Plots LSTM model predictions for each data window in the test set.
\begin{quote}\begin{description}
\sphinxlineitem{Parameters}\begin{itemize}
\item {} 
\sphinxAtStartPar
\sphinxstyleliteralstrong{\sphinxupquote{predictions}} \textendash{} Predictions made by the LSTM model.

\item {} 
\sphinxAtStartPar
\sphinxstyleliteralstrong{\sphinxupquote{y\_test}} \textendash{} Actual test values corresponding to the predictions.

\end{itemize}

\end{description}\end{quote}

\end{fulllineitems}

\index{NAIVE\_plot\_pred() (model\_testing.ModelTest method)@\spxentry{NAIVE\_plot\_pred()}\spxextra{model\_testing.ModelTest method}}

\begin{fulllineitems}
\phantomsection\label{\detokenize{docs/model_testing:model_testing.ModelTest.NAIVE_plot_pred}}
\pysigstartsignatures
\pysiglinewithargsret{\sphinxbfcode{\sphinxupquote{NAIVE\_plot\_pred}}}{\sphinxparam{\DUrole{n}{naive\_predictions}}}{}
\pysigstopsignatures
\sphinxAtStartPar
Plots naive predictions against the test data.
\begin{quote}\begin{description}
\sphinxlineitem{Parameters}
\sphinxAtStartPar
\sphinxstyleliteralstrong{\sphinxupquote{naive\_predictions}} \textendash{} The naive predictions to plot.

\end{description}\end{quote}

\end{fulllineitems}

\index{XGB\_plot\_pred() (model\_testing.ModelTest method)@\spxentry{XGB\_plot\_pred()}\spxextra{model\_testing.ModelTest method}}

\begin{fulllineitems}
\phantomsection\label{\detokenize{docs/model_testing:model_testing.ModelTest.XGB_plot_pred}}
\pysigstartsignatures
\pysiglinewithargsret{\sphinxbfcode{\sphinxupquote{XGB\_plot\_pred}}}{\sphinxparam{\DUrole{n}{test}}\sphinxparamcomma \sphinxparam{\DUrole{n}{predictions}}\sphinxparamcomma \sphinxparam{\DUrole{n}{time\_values}}}{}
\pysigstopsignatures
\sphinxAtStartPar
Plots predictions made by an XGBoost model against the test data.
\begin{quote}\begin{description}
\sphinxlineitem{Parameters}\begin{itemize}
\item {} 
\sphinxAtStartPar
\sphinxstyleliteralstrong{\sphinxupquote{test}} \textendash{} The actual test data.

\item {} 
\sphinxAtStartPar
\sphinxstyleliteralstrong{\sphinxupquote{predictions}} \textendash{} The predictions made by the model.

\item {} 
\sphinxAtStartPar
\sphinxstyleliteralstrong{\sphinxupquote{time\_values}} \textendash{} Time values corresponding to the test data.

\end{itemize}

\end{description}\end{quote}

\end{fulllineitems}

\index{naive\_forecast() (model\_testing.ModelTest method)@\spxentry{naive\_forecast()}\spxextra{model\_testing.ModelTest method}}

\begin{fulllineitems}
\phantomsection\label{\detokenize{docs/model_testing:model_testing.ModelTest.naive_forecast}}
\pysigstartsignatures
\pysiglinewithargsret{\sphinxbfcode{\sphinxupquote{naive\_forecast}}}{\sphinxparam{\DUrole{n}{train}}}{}
\pysigstopsignatures
\sphinxAtStartPar
Performs a naive forecast using the last observed value from the training set.
\begin{quote}\begin{description}
\sphinxlineitem{Parameters}
\sphinxAtStartPar
\sphinxstyleliteralstrong{\sphinxupquote{train}} \textendash{} The training set.

\sphinxlineitem{Returns}
\sphinxAtStartPar
A pandas Series of naive forecasts.

\end{description}\end{quote}

\end{fulllineitems}

\index{naive\_mean\_forecast() (model\_testing.ModelTest method)@\spxentry{naive\_mean\_forecast()}\spxextra{model\_testing.ModelTest method}}

\begin{fulllineitems}
\phantomsection\label{\detokenize{docs/model_testing:model_testing.ModelTest.naive_mean_forecast}}
\pysigstartsignatures
\pysiglinewithargsret{\sphinxbfcode{\sphinxupquote{naive\_mean\_forecast}}}{\sphinxparam{\DUrole{n}{train}}}{}
\pysigstopsignatures
\sphinxAtStartPar
Performs a naive forecast using the mean value of the training set.
\begin{quote}\begin{description}
\sphinxlineitem{Parameters}
\sphinxAtStartPar
\sphinxstyleliteralstrong{\sphinxupquote{train}} \textendash{} The training set.

\sphinxlineitem{Returns}
\sphinxAtStartPar
A pandas Series of naive forecasts using the mean.

\end{description}\end{quote}

\end{fulllineitems}

\index{naive\_seasonal\_forecast() (model\_testing.ModelTest method)@\spxentry{naive\_seasonal\_forecast()}\spxextra{model\_testing.ModelTest method}}

\begin{fulllineitems}
\phantomsection\label{\detokenize{docs/model_testing:model_testing.ModelTest.naive_seasonal_forecast}}
\pysigstartsignatures
\pysiglinewithargsret{\sphinxbfcode{\sphinxupquote{naive\_seasonal\_forecast}}}{\sphinxparam{\DUrole{n}{train}}\sphinxparamcomma \sphinxparam{\DUrole{n}{target\_test}}\sphinxparamcomma \sphinxparam{\DUrole{n}{period}\DUrole{o}{=}\DUrole{default_value}{24}}}{}
\pysigstopsignatures
\sphinxAtStartPar
Performs a seasonal naive forecast using the last observed seasonal cycle.
\begin{quote}\begin{description}
\sphinxlineitem{Parameters}\begin{itemize}
\item {} 
\sphinxAtStartPar
\sphinxstyleliteralstrong{\sphinxupquote{train}} \textendash{} The training set.

\item {} 
\sphinxAtStartPar
\sphinxstyleliteralstrong{\sphinxupquote{target\_test}} \textendash{} The test set.

\item {} 
\sphinxAtStartPar
\sphinxstyleliteralstrong{\sphinxupquote{period}} \textendash{} The seasonal period to consider for the forecast.

\end{itemize}

\sphinxlineitem{Returns}
\sphinxAtStartPar
A pandas Series of naive seasonal forecasts.

\end{description}\end{quote}

\end{fulllineitems}

\index{test\_ARIMA\_model() (model\_testing.ModelTest method)@\spxentry{test\_ARIMA\_model()}\spxextra{model\_testing.ModelTest method}}

\begin{fulllineitems}
\phantomsection\label{\detokenize{docs/model_testing:model_testing.ModelTest.test_ARIMA_model}}
\pysigstartsignatures
\pysiglinewithargsret{\sphinxbfcode{\sphinxupquote{test\_ARIMA\_model}}}{\sphinxparam{\DUrole{n}{last\_index}}\sphinxparamcomma \sphinxparam{\DUrole{n}{steps\_jump}\DUrole{o}{=}\DUrole{default_value}{None}}\sphinxparamcomma \sphinxparam{\DUrole{n}{ol\_refit}\DUrole{o}{=}\DUrole{default_value}{False}}}{}
\pysigstopsignatures
\sphinxAtStartPar
Tests an ARIMA model by performing one step\sphinxhyphen{}ahead predictions and optionally refitting the model.
\begin{quote}\begin{description}
\sphinxlineitem{Parameters}\begin{itemize}
\item {} 
\sphinxAtStartPar
\sphinxstyleliteralstrong{\sphinxupquote{steps\_jump}} \textendash{} Optional parameter to skip steps in the forecasting.

\item {} 
\sphinxAtStartPar
\sphinxstyleliteralstrong{\sphinxupquote{ol\_refit}} \textendash{} Boolean indicating whether to refit the model after each forecast.

\item {} 
\sphinxAtStartPar
\sphinxstyleliteralstrong{\sphinxupquote{last\_index}} \textendash{} index of last training/validation timestep

\end{itemize}

\sphinxlineitem{Returns}
\sphinxAtStartPar
A pandas Series of the predictions.

\end{description}\end{quote}

\end{fulllineitems}

\index{test\_SARIMAX\_model() (model\_testing.ModelTest method)@\spxentry{test\_SARIMAX\_model()}\spxextra{model\_testing.ModelTest method}}

\begin{fulllineitems}
\phantomsection\label{\detokenize{docs/model_testing:model_testing.ModelTest.test_SARIMAX_model}}
\pysigstartsignatures
\pysiglinewithargsret{\sphinxbfcode{\sphinxupquote{test\_SARIMAX\_model}}}{\sphinxparam{\DUrole{n}{last\_index}}\sphinxparamcomma \sphinxparam{\DUrole{n}{steps\_jump}\DUrole{o}{=}\DUrole{default_value}{None}}\sphinxparamcomma \sphinxparam{\DUrole{n}{exog\_test}\DUrole{o}{=}\DUrole{default_value}{None}}\sphinxparamcomma \sphinxparam{\DUrole{n}{ol\_refit}\DUrole{o}{=}\DUrole{default_value}{False}}\sphinxparamcomma \sphinxparam{\DUrole{n}{period}\DUrole{o}{=}\DUrole{default_value}{24}}\sphinxparamcomma \sphinxparam{\DUrole{n}{set\_Fourier}\DUrole{o}{=}\DUrole{default_value}{False}}}{}
\pysigstopsignatures
\sphinxAtStartPar
Tests a SARIMAX model by performing one\sphinxhyphen{}step or multi\sphinxhyphen{}step ahead predictions, optionally using exogenous variables or applying refitting.
\begin{quote}\begin{description}
\sphinxlineitem{Parameters}\begin{itemize}
\item {} 
\sphinxAtStartPar
\sphinxstyleliteralstrong{\sphinxupquote{last\_index}} \textendash{} Index of the last training/validation timestep.

\item {} 
\sphinxAtStartPar
\sphinxstyleliteralstrong{\sphinxupquote{steps\_jump}} \textendash{} Optional parameter to skip steps in the forecasting.

\item {} 
\sphinxAtStartPar
\sphinxstyleliteralstrong{\sphinxupquote{exog\_test}} \textendash{} Optional exogenous variables for the test set.

\item {} 
\sphinxAtStartPar
\sphinxstyleliteralstrong{\sphinxupquote{ol\_refit}} \textendash{} Boolean indicating whether to refit the model after each forecast.

\item {} 
\sphinxAtStartPar
\sphinxstyleliteralstrong{\sphinxupquote{period}} \textendash{} The period for Fourier terms if set\_Fourier is True.

\item {} 
\sphinxAtStartPar
\sphinxstyleliteralstrong{\sphinxupquote{set\_Fourier}} \textendash{} Boolean flag to determine if Fourier terms should be included.

\end{itemize}

\sphinxlineitem{Returns}
\sphinxAtStartPar
A pandas Series of the predictions.

\end{description}\end{quote}

\end{fulllineitems}


\end{fulllineitems}


\sphinxAtStartPar
Class for testing models.

\sphinxstepscope


\section{Performance Measurement}
\label{\detokenize{docs/performance_measurement:performance-measurement}}\label{\detokenize{docs/performance_measurement::doc}}\index{module@\spxentry{module}!performance\_measurement@\spxentry{performance\_measurement}}\index{performance\_measurement@\spxentry{performance\_measurement}!module@\spxentry{module}}\index{PerfMeasure (class in performance\_measurement)@\spxentry{PerfMeasure}\spxextra{class in performance\_measurement}}\phantomsection\label{\detokenize{docs/performance_measurement:module-performance_measurement}}

\begin{fulllineitems}
\phantomsection\label{\detokenize{docs/performance_measurement:performance_measurement.PerfMeasure}}
\pysigstartsignatures
\pysiglinewithargsret{\sphinxbfcode{\sphinxupquote{class\DUrole{w}{ }}}\sphinxcode{\sphinxupquote{performance\_measurement.}}\sphinxbfcode{\sphinxupquote{PerfMeasure}}}{\sphinxparam{\DUrole{n}{model\_type}}\sphinxparamcomma \sphinxparam{\DUrole{n}{model}}\sphinxparamcomma \sphinxparam{\DUrole{n}{test}}\sphinxparamcomma \sphinxparam{\DUrole{n}{target\_column}}\sphinxparamcomma \sphinxparam{\DUrole{n}{forecast\_type}}}{}
\pysigstopsignatures
\sphinxAtStartPar
Bases: \sphinxcode{\sphinxupquote{ModelTest}}

\sphinxAtStartPar
Class that extends ModelTest to provide additional methods for measuring and plotting performance
metrics of forecasting models. It calculates and returns various performance metrics
for a given set of actual and predicted values.
\begin{quote}\begin{description}
\sphinxlineitem{Parameters}\begin{itemize}
\item {} 
\sphinxAtStartPar
\sphinxstyleliteralstrong{\sphinxupquote{model\_type}} \textendash{} The type of model to test (‘ARIMA’, ‘SARIMAX’, etc.).

\item {} 
\sphinxAtStartPar
\sphinxstyleliteralstrong{\sphinxupquote{model}} \textendash{} The model object to be tested.

\item {} 
\sphinxAtStartPar
\sphinxstyleliteralstrong{\sphinxupquote{test}} \textendash{} The test set.

\item {} 
\sphinxAtStartPar
\sphinxstyleliteralstrong{\sphinxupquote{target\_column}} \textendash{} The target column in the dataset.

\item {} 
\sphinxAtStartPar
\sphinxstyleliteralstrong{\sphinxupquote{forecast\_type}} \textendash{} The type of forecasting to be performed (‘ol\sphinxhyphen{}one’, etc.).

\end{itemize}

\end{description}\end{quote}
\index{get\_performance\_metrics() (performance\_measurement.PerfMeasure method)@\spxentry{get\_performance\_metrics()}\spxextra{performance\_measurement.PerfMeasure method}}

\begin{fulllineitems}
\phantomsection\label{\detokenize{docs/performance_measurement:performance_measurement.PerfMeasure.get_performance_metrics}}
\pysigstartsignatures
\pysiglinewithargsret{\sphinxbfcode{\sphinxupquote{get\_performance\_metrics}}}{\sphinxparam{\DUrole{n}{test}}\sphinxparamcomma \sphinxparam{\DUrole{n}{predictions}}\sphinxparamcomma \sphinxparam{\DUrole{n}{naive}\DUrole{o}{=}\DUrole{default_value}{False}}}{}
\pysigstopsignatures
\sphinxAtStartPar
Calculates a set of performance metrics for model evaluation.
\begin{quote}\begin{description}
\sphinxlineitem{Parameters}\begin{itemize}
\item {} 
\sphinxAtStartPar
\sphinxstyleliteralstrong{\sphinxupquote{test}} \textendash{} The actual test data.

\item {} 
\sphinxAtStartPar
\sphinxstyleliteralstrong{\sphinxupquote{predictions}} \textendash{} Predicted values by the model.

\item {} 
\sphinxAtStartPar
\sphinxstyleliteralstrong{\sphinxupquote{naive}} \textendash{} Boolean flag to indicate if the naive predictions should be considered.

\end{itemize}

\sphinxlineitem{Returns}
\sphinxAtStartPar
A dictionary of performance metrics including MSE, RMSE, MAPE, MSPE, MAE, and R\sphinxhyphen{}squared.

\end{description}\end{quote}

\end{fulllineitems}


\end{fulllineitems}


\sphinxAtStartPar
Class for measuring the performance of forecasting models.

\sphinxstepscope


\section{Time Series Analysis}
\label{\detokenize{docs/time_series_analysis:time-series-analysis}}\label{\detokenize{docs/time_series_analysis::doc}}\index{module@\spxentry{module}!time\_series\_analysis@\spxentry{time\_series\_analysis}}\index{time\_series\_analysis@\spxentry{time\_series\_analysis}!module@\spxentry{module}}\index{ARIMA\_optimizer() (in module time\_series\_analysis)@\spxentry{ARIMA\_optimizer()}\spxextra{in module time\_series\_analysis}}\phantomsection\label{\detokenize{docs/time_series_analysis:module-time_series_analysis}}

\begin{fulllineitems}
\phantomsection\label{\detokenize{docs/time_series_analysis:time_series_analysis.ARIMA_optimizer}}
\pysigstartsignatures
\pysiglinewithargsret{\sphinxcode{\sphinxupquote{time\_series\_analysis.}}\sphinxbfcode{\sphinxupquote{ARIMA\_optimizer}}}{\sphinxparam{\DUrole{n}{train}}\sphinxparamcomma \sphinxparam{\DUrole{n}{target\_column}\DUrole{o}{=}\DUrole{default_value}{None}}\sphinxparamcomma \sphinxparam{\DUrole{n}{verbose}\DUrole{o}{=}\DUrole{default_value}{False}}}{}
\pysigstopsignatures
\sphinxAtStartPar
Determines the optimal parameters for an ARIMA model based on the Akaike Information Criterion (AIC).
\begin{quote}\begin{description}
\sphinxlineitem{Parameters}\begin{itemize}
\item {} 
\sphinxAtStartPar
\sphinxstyleliteralstrong{\sphinxupquote{train}} \textendash{} The training dataset.

\item {} 
\sphinxAtStartPar
\sphinxstyleliteralstrong{\sphinxupquote{target\_column}} \textendash{} The target column in the dataset that needs to be forecasted.

\item {} 
\sphinxAtStartPar
\sphinxstyleliteralstrong{\sphinxupquote{verbose}} \textendash{} If set to True, prints the process of optimization.

\end{itemize}

\sphinxlineitem{Returns}
\sphinxAtStartPar
The best (p, d, q) order for the ARIMA model.

\end{description}\end{quote}

\end{fulllineitems}

\index{SARIMAX\_optimizer() (in module time\_series\_analysis)@\spxentry{SARIMAX\_optimizer()}\spxextra{in module time\_series\_analysis}}

\begin{fulllineitems}
\phantomsection\label{\detokenize{docs/time_series_analysis:time_series_analysis.SARIMAX_optimizer}}
\pysigstartsignatures
\pysiglinewithargsret{\sphinxcode{\sphinxupquote{time\_series\_analysis.}}\sphinxbfcode{\sphinxupquote{SARIMAX\_optimizer}}}{\sphinxparam{\DUrole{n}{train}}\sphinxparamcomma \sphinxparam{\DUrole{n}{target\_column}\DUrole{o}{=}\DUrole{default_value}{None}}\sphinxparamcomma \sphinxparam{\DUrole{n}{period}\DUrole{o}{=}\DUrole{default_value}{None}}\sphinxparamcomma \sphinxparam{\DUrole{n}{exog}\DUrole{o}{=}\DUrole{default_value}{None}}\sphinxparamcomma \sphinxparam{\DUrole{n}{verbose}\DUrole{o}{=}\DUrole{default_value}{False}}}{}
\pysigstopsignatures
\sphinxAtStartPar
Identifies the optimal parameters for a SARIMAX model.
\begin{quote}\begin{description}
\sphinxlineitem{Parameters}\begin{itemize}
\item {} 
\sphinxAtStartPar
\sphinxstyleliteralstrong{\sphinxupquote{train}} \textendash{} The training dataset.

\item {} 
\sphinxAtStartPar
\sphinxstyleliteralstrong{\sphinxupquote{target\_column}} \textendash{} The target column in the dataset.

\item {} 
\sphinxAtStartPar
\sphinxstyleliteralstrong{\sphinxupquote{period}} \textendash{} The seasonal period of the dataset.

\item {} 
\sphinxAtStartPar
\sphinxstyleliteralstrong{\sphinxupquote{exog}} \textendash{} The exogenous variables included in the model.

\item {} 
\sphinxAtStartPar
\sphinxstyleliteralstrong{\sphinxupquote{verbose}} \textendash{} Controls the output of the optimization process.

\end{itemize}

\sphinxlineitem{Returns}
\sphinxAtStartPar
The best (p, d, q, P, D, Q) parameters for the SARIMAX model.

\end{description}\end{quote}

\end{fulllineitems}

\index{adf\_test() (in module time\_series\_analysis)@\spxentry{adf\_test()}\spxextra{in module time\_series\_analysis}}

\begin{fulllineitems}
\phantomsection\label{\detokenize{docs/time_series_analysis:time_series_analysis.adf_test}}
\pysigstartsignatures
\pysiglinewithargsret{\sphinxcode{\sphinxupquote{time\_series\_analysis.}}\sphinxbfcode{\sphinxupquote{adf\_test}}}{\sphinxparam{\DUrole{n}{df}}\sphinxparamcomma \sphinxparam{\DUrole{n}{alpha}\DUrole{o}{=}\DUrole{default_value}{0.05}}\sphinxparamcomma \sphinxparam{\DUrole{n}{verbose}\DUrole{o}{=}\DUrole{default_value}{False}}}{}
\pysigstopsignatures
\sphinxAtStartPar
Performs the Augmented Dickey\sphinxhyphen{}Fuller test to determine if a series is stationary and provides detailed output.
\begin{quote}\begin{description}
\sphinxlineitem{Parameters}\begin{itemize}
\item {} 
\sphinxAtStartPar
\sphinxstyleliteralstrong{\sphinxupquote{df}} \textendash{} The time series data as a DataFrame.

\item {} 
\sphinxAtStartPar
\sphinxstyleliteralstrong{\sphinxupquote{alpha}} \textendash{} The significance level for the test to determine stationarity.

\item {} 
\sphinxAtStartPar
\sphinxstyleliteralstrong{\sphinxupquote{verbose}} \textendash{} Boolean flag that determines whether to print detailed results.

\end{itemize}

\sphinxlineitem{Returns}
\sphinxAtStartPar
The number of differences needed to make the series stationary.

\end{description}\end{quote}

\end{fulllineitems}

\index{conditional\_print() (in module time\_series\_analysis)@\spxentry{conditional\_print()}\spxextra{in module time\_series\_analysis}}

\begin{fulllineitems}
\phantomsection\label{\detokenize{docs/time_series_analysis:time_series_analysis.conditional_print}}
\pysigstartsignatures
\pysiglinewithargsret{\sphinxcode{\sphinxupquote{time\_series\_analysis.}}\sphinxbfcode{\sphinxupquote{conditional\_print}}}{\sphinxparam{\DUrole{n}{verbose}}\sphinxparamcomma \sphinxparam{\DUrole{o}{*}\DUrole{n}{args}}\sphinxparamcomma \sphinxparam{\DUrole{o}{**}\DUrole{n}{kwargs}}}{}
\pysigstopsignatures
\sphinxAtStartPar
Prints messages conditionally based on a verbosity flag.
\begin{quote}\begin{description}
\sphinxlineitem{Parameters}\begin{itemize}
\item {} 
\sphinxAtStartPar
\sphinxstyleliteralstrong{\sphinxupquote{verbose}} \textendash{} Boolean flag indicating whether to print messages.

\item {} 
\sphinxAtStartPar
\sphinxstyleliteralstrong{\sphinxupquote{args}} \textendash{} Arguments to be printed.

\item {} 
\sphinxAtStartPar
\sphinxstyleliteralstrong{\sphinxupquote{kwargs}} \textendash{} Keyword arguments to be printed.

\end{itemize}

\end{description}\end{quote}

\end{fulllineitems}

\index{ljung\_box\_test() (in module time\_series\_analysis)@\spxentry{ljung\_box\_test()}\spxextra{in module time\_series\_analysis}}

\begin{fulllineitems}
\phantomsection\label{\detokenize{docs/time_series_analysis:time_series_analysis.ljung_box_test}}
\pysigstartsignatures
\pysiglinewithargsret{\sphinxcode{\sphinxupquote{time\_series\_analysis.}}\sphinxbfcode{\sphinxupquote{ljung\_box\_test}}}{\sphinxparam{\DUrole{n}{model}}}{}
\pysigstopsignatures
\sphinxAtStartPar
Conducts the Ljung\sphinxhyphen{}Box test on the residuals of a fitted time series model to check for autocorrelation.
\begin{quote}\begin{description}
\sphinxlineitem{Parameters}
\sphinxAtStartPar
\sphinxstyleliteralstrong{\sphinxupquote{model}} \textendash{} The time series model after fitting to the data.

\end{description}\end{quote}

\end{fulllineitems}

\index{multiple\_STL() (in module time\_series\_analysis)@\spxentry{multiple\_STL()}\spxextra{in module time\_series\_analysis}}

\begin{fulllineitems}
\phantomsection\label{\detokenize{docs/time_series_analysis:time_series_analysis.multiple_STL}}
\pysigstartsignatures
\pysiglinewithargsret{\sphinxcode{\sphinxupquote{time\_series\_analysis.}}\sphinxbfcode{\sphinxupquote{multiple\_STL}}}{\sphinxparam{\DUrole{n}{dataframe}}\sphinxparamcomma \sphinxparam{\DUrole{n}{target\_column}}}{}
\pysigstopsignatures
\sphinxAtStartPar
Performs multiple seasonal decomposition using STL on specified periods.
\begin{quote}\begin{description}
\sphinxlineitem{Parameters}\begin{itemize}
\item {} 
\sphinxAtStartPar
\sphinxstyleliteralstrong{\sphinxupquote{dataframe}} \textendash{} The DataFrame containing the time series data.

\item {} 
\sphinxAtStartPar
\sphinxstyleliteralstrong{\sphinxupquote{target\_column}} \textendash{} The column in the DataFrame to be decomposed.

\end{itemize}

\end{description}\end{quote}

\end{fulllineitems}

\index{optimize\_ARIMA() (in module time\_series\_analysis)@\spxentry{optimize\_ARIMA()}\spxextra{in module time\_series\_analysis}}

\begin{fulllineitems}
\phantomsection\label{\detokenize{docs/time_series_analysis:time_series_analysis.optimize_ARIMA}}
\pysigstartsignatures
\pysiglinewithargsret{\sphinxcode{\sphinxupquote{time\_series\_analysis.}}\sphinxbfcode{\sphinxupquote{optimize\_ARIMA}}}{\sphinxparam{\DUrole{n}{endog}}\sphinxparamcomma \sphinxparam{\DUrole{n}{order\_list}}}{}
\pysigstopsignatures
\sphinxAtStartPar
Optimizes ARIMA parameters by iterating over a list of (p, d, q) combinations to find the lowest AIC.
\begin{quote}\begin{description}
\sphinxlineitem{Parameters}\begin{itemize}
\item {} 
\sphinxAtStartPar
\sphinxstyleliteralstrong{\sphinxupquote{endog}} \textendash{} The endogenous variable.

\item {} 
\sphinxAtStartPar
\sphinxstyleliteralstrong{\sphinxupquote{order\_list}} \textendash{} A list of (p, d, q) tuples representing different ARIMA configurations to test.

\end{itemize}

\sphinxlineitem{Returns}
\sphinxAtStartPar
A DataFrame containing the AIC scores for each parameter combination.

\end{description}\end{quote}

\end{fulllineitems}

\index{optimize\_SARIMAX() (in module time\_series\_analysis)@\spxentry{optimize\_SARIMAX()}\spxextra{in module time\_series\_analysis}}

\begin{fulllineitems}
\phantomsection\label{\detokenize{docs/time_series_analysis:time_series_analysis.optimize_SARIMAX}}
\pysigstartsignatures
\pysiglinewithargsret{\sphinxcode{\sphinxupquote{time\_series\_analysis.}}\sphinxbfcode{\sphinxupquote{optimize\_SARIMAX}}}{\sphinxparam{\DUrole{n}{endog}}\sphinxparamcomma \sphinxparam{\DUrole{n}{order\_list}}\sphinxparamcomma \sphinxparam{\DUrole{n}{s}}\sphinxparamcomma \sphinxparam{\DUrole{n}{exog}\DUrole{o}{=}\DUrole{default_value}{None}}}{}
\pysigstopsignatures
\sphinxAtStartPar
Optimizes SARIMAX parameters by testing various combinations and selecting the one with the lowest AIC.
\begin{quote}\begin{description}
\sphinxlineitem{Parameters}\begin{itemize}
\item {} 
\sphinxAtStartPar
\sphinxstyleliteralstrong{\sphinxupquote{endog}} \textendash{} The dependent variable.

\item {} 
\sphinxAtStartPar
\sphinxstyleliteralstrong{\sphinxupquote{order\_list}} \textendash{} A list of order tuples (p, d, q, P, D, Q) for the SARIMAX.

\item {} 
\sphinxAtStartPar
\sphinxstyleliteralstrong{\sphinxupquote{s}} \textendash{} The seasonal period of the model.

\item {} 
\sphinxAtStartPar
\sphinxstyleliteralstrong{\sphinxupquote{exog}} \textendash{} Optional exogenous variables.

\end{itemize}

\sphinxlineitem{Returns}
\sphinxAtStartPar
A DataFrame with the results of the parameter testing.

\end{description}\end{quote}

\end{fulllineitems}

\index{prepare\_seasonal\_sets() (in module time\_series\_analysis)@\spxentry{prepare\_seasonal\_sets()}\spxextra{in module time\_series\_analysis}}

\begin{fulllineitems}
\phantomsection\label{\detokenize{docs/time_series_analysis:time_series_analysis.prepare_seasonal_sets}}
\pysigstartsignatures
\pysiglinewithargsret{\sphinxcode{\sphinxupquote{time\_series\_analysis.}}\sphinxbfcode{\sphinxupquote{prepare\_seasonal\_sets}}}{\sphinxparam{\DUrole{n}{train}}\sphinxparamcomma \sphinxparam{\DUrole{n}{valid}}\sphinxparamcomma \sphinxparam{\DUrole{n}{test}}\sphinxparamcomma \sphinxparam{\DUrole{n}{target\_column}}\sphinxparamcomma \sphinxparam{\DUrole{n}{period}}}{}
\pysigstopsignatures
\sphinxAtStartPar
Decomposes the datasets into seasonal and residual components based on the specified period.
\begin{quote}\begin{description}
\sphinxlineitem{Parameters}\begin{itemize}
\item {} 
\sphinxAtStartPar
\sphinxstyleliteralstrong{\sphinxupquote{train}} \textendash{} Training dataset.

\item {} 
\sphinxAtStartPar
\sphinxstyleliteralstrong{\sphinxupquote{valid}} \textendash{} Validation dataset.

\item {} 
\sphinxAtStartPar
\sphinxstyleliteralstrong{\sphinxupquote{test}} \textendash{} Test dataset.

\item {} 
\sphinxAtStartPar
\sphinxstyleliteralstrong{\sphinxupquote{target\_column}} \textendash{} The target column in the datasets.

\item {} 
\sphinxAtStartPar
\sphinxstyleliteralstrong{\sphinxupquote{period}} \textendash{} The period for seasonal decomposition.

\end{itemize}

\sphinxlineitem{Returns}
\sphinxAtStartPar
Decomposed training, validation, and test datasets.

\end{description}\end{quote}

\end{fulllineitems}

\index{time\_s\_analysis() (in module time\_series\_analysis)@\spxentry{time\_s\_analysis()}\spxextra{in module time\_series\_analysis}}

\begin{fulllineitems}
\phantomsection\label{\detokenize{docs/time_series_analysis:time_series_analysis.time_s_analysis}}
\pysigstartsignatures
\pysiglinewithargsret{\sphinxcode{\sphinxupquote{time\_series\_analysis.}}\sphinxbfcode{\sphinxupquote{time\_s\_analysis}}}{\sphinxparam{\DUrole{n}{df}}\sphinxparamcomma \sphinxparam{\DUrole{n}{target\_column}}\sphinxparamcomma \sphinxparam{\DUrole{n}{seasonal\_period}}}{}
\pysigstopsignatures
\sphinxAtStartPar
Performs a comprehensive time series analysis including plotting, stationarity testing, and decomposition.
\begin{quote}\begin{description}
\sphinxlineitem{Parameters}\begin{itemize}
\item {} 
\sphinxAtStartPar
\sphinxstyleliteralstrong{\sphinxupquote{df}} \textendash{} The DataFrame containing the time series data.

\item {} 
\sphinxAtStartPar
\sphinxstyleliteralstrong{\sphinxupquote{target\_column}} \textendash{} The column in the DataFrame representing the time series to analyze.

\item {} 
\sphinxAtStartPar
\sphinxstyleliteralstrong{\sphinxupquote{seasonal\_period}} \textendash{} The period to consider for seasonal decomposition and autocorrelation analysis.

\end{itemize}

\end{description}\end{quote}

\end{fulllineitems}


\sphinxAtStartPar
Functions for time series analysis, statistic tests and optimizing statistical models.

\sphinxstepscope


\section{Utilities}
\label{\detokenize{docs/utilities:module-utilities}}\label{\detokenize{docs/utilities:utilities}}\label{\detokenize{docs/utilities::doc}}\index{module@\spxentry{module}!utilities@\spxentry{utilities}}\index{utilities@\spxentry{utilities}!module@\spxentry{module}}\index{conditional\_print() (in module utilities)@\spxentry{conditional\_print()}\spxextra{in module utilities}}

\begin{fulllineitems}
\phantomsection\label{\detokenize{docs/utilities:utilities.conditional_print}}
\pysigstartsignatures
\pysiglinewithargsret{\sphinxcode{\sphinxupquote{utilities.}}\sphinxbfcode{\sphinxupquote{conditional\_print}}}{\sphinxparam{\DUrole{n}{verbose}}\sphinxparamcomma \sphinxparam{\DUrole{o}{*}\DUrole{n}{args}}\sphinxparamcomma \sphinxparam{\DUrole{o}{**}\DUrole{n}{kwargs}}}{}
\pysigstopsignatures
\sphinxAtStartPar
Prints provided arguments if the verbose flag is set to True.
\begin{quote}\begin{description}
\sphinxlineitem{Parameters}\begin{itemize}
\item {} 
\sphinxAtStartPar
\sphinxstyleliteralstrong{\sphinxupquote{verbose}} \textendash{} Boolean, controlling whether to print.

\item {} 
\sphinxAtStartPar
\sphinxstyleliteralstrong{\sphinxupquote{args}} \textendash{} Arguments to be printed.

\item {} 
\sphinxAtStartPar
\sphinxstyleliteralstrong{\sphinxupquote{kwargs}} \textendash{} Keyword arguments to be printed.

\end{itemize}

\end{description}\end{quote}

\end{fulllineitems}

\index{load\_trained\_model() (in module utilities)@\spxentry{load\_trained\_model()}\spxextra{in module utilities}}

\begin{fulllineitems}
\phantomsection\label{\detokenize{docs/utilities:utilities.load_trained_model}}
\pysigstartsignatures
\pysiglinewithargsret{\sphinxcode{\sphinxupquote{utilities.}}\sphinxbfcode{\sphinxupquote{load\_trained\_model}}}{\sphinxparam{\DUrole{n}{model\_type}}\sphinxparamcomma \sphinxparam{\DUrole{n}{folder\_name}}}{}
\pysigstopsignatures
\sphinxAtStartPar
Loads a trained model and its configuration from the selected directory.
\begin{quote}\begin{description}
\sphinxlineitem{Parameters}\begin{itemize}
\item {} 
\sphinxAtStartPar
\sphinxstyleliteralstrong{\sphinxupquote{model\_type}} \textendash{} Type of the model to load (‘ARIMA’, ‘SARIMAX’, etc.).

\item {} 
\sphinxAtStartPar
\sphinxstyleliteralstrong{\sphinxupquote{folder\_name}} \textendash{} Directory from which the model and its details will be loaded.

\end{itemize}

\sphinxlineitem{Returns}
\sphinxAtStartPar
A tuple containing the loaded model and its order (if applicable).

\end{description}\end{quote}

\end{fulllineitems}

\index{naive\_forecast() (in module utilities)@\spxentry{naive\_forecast()}\spxextra{in module utilities}}

\begin{fulllineitems}
\phantomsection\label{\detokenize{docs/utilities:utilities.naive_forecast}}
\pysigstartsignatures
\pysiglinewithargsret{\sphinxcode{\sphinxupquote{utilities.}}\sphinxbfcode{\sphinxupquote{naive\_forecast}}}{\sphinxparam{\DUrole{n}{train}}\sphinxparamcomma \sphinxparam{\DUrole{n}{test}}\sphinxparamcomma \sphinxparam{\DUrole{n}{target\_column}}\sphinxparamcomma \sphinxparam{\DUrole{n}{steps\_ahead}\DUrole{o}{=}\DUrole{default_value}{None}}}{}
\pysigstopsignatures
\sphinxAtStartPar
Performs a naive forecast using the last observed value from the training set.
\begin{quote}\begin{description}
\sphinxlineitem{Parameters}\begin{itemize}
\item {} 
\sphinxAtStartPar
\sphinxstyleliteralstrong{\sphinxupquote{train}} \textendash{} The training set.

\item {} 
\sphinxAtStartPar
\sphinxstyleliteralstrong{\sphinxupquote{test}} \textendash{} The test set.

\item {} 
\sphinxAtStartPar
\sphinxstyleliteralstrong{\sphinxupquote{target\_column}} \textendash{} Column name for the target variable.

\item {} 
\sphinxAtStartPar
\sphinxstyleliteralstrong{\sphinxupquote{steps\_ahead}} \textendash{} Number of steps to forecast ahead, if not set, matches the test set length.

\end{itemize}

\sphinxlineitem{Returns}
\sphinxAtStartPar
A pandas Series of naive forecasts.

\end{description}\end{quote}

\end{fulllineitems}

\index{naive\_seasonal\_forecast() (in module utilities)@\spxentry{naive\_seasonal\_forecast()}\spxextra{in module utilities}}

\begin{fulllineitems}
\phantomsection\label{\detokenize{docs/utilities:utilities.naive_seasonal_forecast}}
\pysigstartsignatures
\pysiglinewithargsret{\sphinxcode{\sphinxupquote{utilities.}}\sphinxbfcode{\sphinxupquote{naive\_seasonal\_forecast}}}{\sphinxparam{\DUrole{n}{train}}\sphinxparamcomma \sphinxparam{\DUrole{n}{test}}\sphinxparamcomma \sphinxparam{\DUrole{n}{target\_column}}\sphinxparamcomma \sphinxparam{\DUrole{n}{steps\_ahead}\DUrole{o}{=}\DUrole{default_value}{None}}\sphinxparamcomma \sphinxparam{\DUrole{n}{period}\DUrole{o}{=}\DUrole{default_value}{24}}}{}
\pysigstopsignatures
\sphinxAtStartPar
Performs a seasonal naive forecast using the last observed seasonal cycle.
\begin{quote}\begin{description}
\sphinxlineitem{Parameters}\begin{itemize}
\item {} 
\sphinxAtStartPar
\sphinxstyleliteralstrong{\sphinxupquote{train}} \textendash{} The training set.

\item {} 
\sphinxAtStartPar
\sphinxstyleliteralstrong{\sphinxupquote{test}} \textendash{} The test set.

\item {} 
\sphinxAtStartPar
\sphinxstyleliteralstrong{\sphinxupquote{target\_column}} \textendash{} Column name for the target variable.

\item {} 
\sphinxAtStartPar
\sphinxstyleliteralstrong{\sphinxupquote{steps\_ahead}} \textendash{} Number of steps to forecast ahead, if not set, matches the test set length.

\item {} 
\sphinxAtStartPar
\sphinxstyleliteralstrong{\sphinxupquote{period}} \textendash{} The seasonal period to consider for the forecast.

\end{itemize}

\sphinxlineitem{Returns}
\sphinxAtStartPar
A pandas Series of naive seasonal forecasts.

\end{description}\end{quote}

\end{fulllineitems}

\index{save\_buffer() (in module utilities)@\spxentry{save\_buffer()}\spxextra{in module utilities}}

\begin{fulllineitems}
\phantomsection\label{\detokenize{docs/utilities:utilities.save_buffer}}
\pysigstartsignatures
\pysiglinewithargsret{\sphinxcode{\sphinxupquote{utilities.}}\sphinxbfcode{\sphinxupquote{save\_buffer}}}{\sphinxparam{\DUrole{n}{folder\_path}}\sphinxparamcomma \sphinxparam{\DUrole{n}{df}}\sphinxparamcomma \sphinxparam{\DUrole{n}{target\_column}}\sphinxparamcomma \sphinxparam{\DUrole{n}{size}\DUrole{o}{=}\DUrole{default_value}{20}}\sphinxparamcomma \sphinxparam{\DUrole{n}{file\_name}\DUrole{o}{=}\DUrole{default_value}{\textquotesingle{}buffer.json\textquotesingle{}}}}{}
\pysigstopsignatures
\sphinxAtStartPar
Saves a buffer of the latest data points to a JSON file.
\begin{quote}\begin{description}
\sphinxlineitem{Parameters}\begin{itemize}
\item {} 
\sphinxAtStartPar
\sphinxstyleliteralstrong{\sphinxupquote{folder\_path}} \textendash{} Directory path where the file will be saved.

\item {} 
\sphinxAtStartPar
\sphinxstyleliteralstrong{\sphinxupquote{df}} \textendash{} DataFrame from which data will be extracted.

\item {} 
\sphinxAtStartPar
\sphinxstyleliteralstrong{\sphinxupquote{target\_column}} \textendash{} Column whose data is to be saved.

\item {} 
\sphinxAtStartPar
\sphinxstyleliteralstrong{\sphinxupquote{size}} \textendash{} Number of rows to save from the end of the DataFrame.

\item {} 
\sphinxAtStartPar
\sphinxstyleliteralstrong{\sphinxupquote{file\_name}} \textendash{} Name of the file to save the data in.

\end{itemize}

\end{description}\end{quote}

\end{fulllineitems}

\index{save\_data() (in module utilities)@\spxentry{save\_data()}\spxextra{in module utilities}}

\begin{fulllineitems}
\phantomsection\label{\detokenize{docs/utilities:utilities.save_data}}
\pysigstartsignatures
\pysiglinewithargsret{\sphinxcode{\sphinxupquote{utilities.}}\sphinxbfcode{\sphinxupquote{save\_data}}}{\sphinxparam{\DUrole{n}{save\_mode}}\sphinxparamcomma \sphinxparam{\DUrole{n}{validation}}\sphinxparamcomma \sphinxparam{\DUrole{n}{path}}\sphinxparamcomma \sphinxparam{\DUrole{n}{model\_type}}\sphinxparamcomma \sphinxparam{\DUrole{n}{model}}\sphinxparamcomma \sphinxparam{\DUrole{n}{dataset}}\sphinxparamcomma \sphinxparam{\DUrole{n}{performance}\DUrole{o}{=}\DUrole{default_value}{None}}\sphinxparamcomma \sphinxparam{\DUrole{n}{naive\_performance}\DUrole{o}{=}\DUrole{default_value}{None}}\sphinxparamcomma \sphinxparam{\DUrole{n}{best\_order}\DUrole{o}{=}\DUrole{default_value}{None}}\sphinxparamcomma \sphinxparam{\DUrole{n}{end\_index}\DUrole{o}{=}\DUrole{default_value}{None}}\sphinxparamcomma \sphinxparam{\DUrole{n}{valid\_metrics}\DUrole{o}{=}\DUrole{default_value}{None}}}{}
\pysigstopsignatures
\sphinxAtStartPar
Saves various types of data to files based on the specified mode.
\begin{quote}\begin{description}
\sphinxlineitem{Parameters}\begin{itemize}
\item {} 
\sphinxAtStartPar
\sphinxstyleliteralstrong{\sphinxupquote{save\_mode}} \textendash{} String, ‘training’ or ‘test’, specifying the type of data to save.

\item {} 
\sphinxAtStartPar
\sphinxstyleliteralstrong{\sphinxupquote{validation}} \textendash{} Boolean, indicates if validation metrics should be saved.

\item {} 
\sphinxAtStartPar
\sphinxstyleliteralstrong{\sphinxupquote{path}} \textendash{} Path where the data will be saved.

\item {} 
\sphinxAtStartPar
\sphinxstyleliteralstrong{\sphinxupquote{model\_type}} \textendash{} Type of model used.

\item {} 
\sphinxAtStartPar
\sphinxstyleliteralstrong{\sphinxupquote{model}} \textendash{} Model object to be saved.

\item {} 
\sphinxAtStartPar
\sphinxstyleliteralstrong{\sphinxupquote{dataset}} \textendash{} Name of the dataset used.

\item {} 
\sphinxAtStartPar
\sphinxstyleliteralstrong{\sphinxupquote{performance}} \textendash{} model performance metrics to be saved.

\item {} 
\sphinxAtStartPar
\sphinxstyleliteralstrong{\sphinxupquote{naive\_performance}} \textendash{} naive model performance metrics to be saved.

\item {} 
\sphinxAtStartPar
\sphinxstyleliteralstrong{\sphinxupquote{best\_order}} \textendash{} best model order to be saved.

\item {} 
\sphinxAtStartPar
\sphinxstyleliteralstrong{\sphinxupquote{end\_index}} \textendash{} index of the last training point.

\item {} 
\sphinxAtStartPar
\sphinxstyleliteralstrong{\sphinxupquote{valid\_metrics}} \textendash{} validation metrics to be saved.

\end{itemize}

\end{description}\end{quote}

\end{fulllineitems}


\sphinxAtStartPar
Functions for loading or saving models and training data.


\chapter{Appendix}
\label{\detokenize{index:appendix}}
\sphinxAtStartPar
Here are presented all the parameters that can be given to the argument parser, specifying their function.

\sphinxstepscope


\section{Parser Arguments}
\label{\detokenize{docs/parser_arguments:parser-arguments}}\label{\detokenize{docs/parser_arguments::doc}}
\sphinxAtStartPar
The command\sphinxhyphen{}line parser manages the framework settings, allowing different implementations and uses of the models.
It provides many options to customize time series analysis or training and testing of models.


\subsection{Parameters}
\label{\detokenize{docs/parser_arguments:parameters}}\begin{description}
\sphinxlineitem{\sphinxstylestrong{\textendash{}verbose}}
\sphinxAtStartPar
Minimizes the additional information provided during the program’s execution if specified. Default: \sphinxcode{\sphinxupquote{False}}.

\sphinxlineitem{\sphinxstylestrong{\textendash{}ts\_analysis}}
\sphinxAtStartPar
If \sphinxcode{\sphinxupquote{True}}, performs an analysis on the time series. Default: \sphinxcode{\sphinxupquote{False}}.

\sphinxlineitem{\sphinxstylestrong{\textendash{}run\_mode}}
\sphinxAtStartPar
Specifies the running mode, which must be one of ‘training’, ‘testing’, ‘both’, or ‘fine tuning’. This parameter is required.

\sphinxlineitem{\sphinxstylestrong{\textendash{}dataset\_path}}
\sphinxAtStartPar
Specifies the file path to the dataset. This parameter is required.

\sphinxlineitem{\sphinxstylestrong{\textendash{}date\_list}}
\sphinxAtStartPar
If the “\textendash{}validation” argument is True, provides start and end dates for training, validation, and test sets, otherwise gives start and end dates for training and test.
In test\sphinxhyphen{}only mode, only the first two dates are used as test set start and end.

\sphinxlineitem{\sphinxstylestrong{\textendash{}seasonal\_split}}
\sphinxAtStartPar
If \sphinxcode{\sphinxupquote{True}}, adjusts the data split to account for seasonality. Default: \sphinxcode{\sphinxupquote{False}}.

\sphinxlineitem{\sphinxstylestrong{\textendash{}train\_size}}
\sphinxAtStartPar
Sets the proportion of data to be used for training. Default: \sphinxcode{\sphinxupquote{0.7}}.

\sphinxlineitem{\sphinxstylestrong{\textendash{}val\_size}}
\sphinxAtStartPar
Sets the proportion of data to be used for validation. Default: \sphinxcode{\sphinxupquote{0.2}}.

\sphinxlineitem{\sphinxstylestrong{\textendash{}test\_size}}
\sphinxAtStartPar
Sets the proportion of data to be used for testing. Default: \sphinxcode{\sphinxupquote{0.1}}.

\sphinxlineitem{\sphinxstylestrong{\textendash{}scaling}}
\sphinxAtStartPar
If \sphinxcode{\sphinxupquote{True}}, scales the data. This parameter is required.

\sphinxlineitem{\sphinxstylestrong{\textendash{}validation}}
\sphinxAtStartPar
If \sphinxcode{\sphinxupquote{True}}, includes validation set creation in the data preparation process (not applicable for ARIMA\sphinxhyphen{}SARIMAX models). Default: \sphinxcode{\sphinxupquote{False}}.

\sphinxlineitem{\sphinxstylestrong{\textendash{}target\_column}}
\sphinxAtStartPar
Specifies the column name to be used as the target variable for forecasting. This parameter is required.

\sphinxlineitem{\sphinxstylestrong{\textendash{}time\_column\_index}}
\sphinxAtStartPar
Specifies the index of the timestamp column in the dataset. Default: \sphinxcode{\sphinxupquote{0}}.

\sphinxlineitem{\sphinxstylestrong{\textendash{}model\_type}}
\sphinxAtStartPar
Specifies the type of model to be used. Options include ‘ARIMA’, ‘SARIMAX’, ‘PROPHET’, ‘CONV’, ‘LSTM’, ‘CNN\_LSTM’. This parameter is required.

\sphinxlineitem{\sphinxstylestrong{\textendash{}forecast\_type}}
\sphinxAtStartPar
Specifies the type of forecast; options are ‘ol\sphinxhyphen{}multi’, ‘ol\sphinxhyphen{}one’, ‘cl\sphinxhyphen{}multi’. Not necessary for ‘PROPHET’.

\sphinxlineitem{\sphinxstylestrong{\textendash{}steps\_ahead}}
\sphinxAtStartPar
Defines the number of time steps to forecast ahead. Default: \sphinxcode{\sphinxupquote{10}}.

\sphinxlineitem{\sphinxstylestrong{\textendash{}steps\_jump}}
\sphinxAtStartPar
Specifies the number of time steps to skip. Default: \sphinxcode{\sphinxupquote{50}}.

\sphinxlineitem{\sphinxstylestrong{\textendash{}exog}}
\sphinxAtStartPar
Specifies exogenous columns for the SARIMAX model. Accepts multiple values.

\sphinxlineitem{\sphinxstylestrong{\textendash{}period}}
\sphinxAtStartPar
Defines the seasonality period for the SARIMAX model. Default: \sphinxcode{\sphinxupquote{24}}.

\sphinxlineitem{\sphinxstylestrong{\textendash{}seasonal\_model}}
\sphinxAtStartPar
If \sphinxcode{\sphinxupquote{True}}, performs a seasonal decomposition, and the seasonal component is fed into the LSTM model.

\sphinxlineitem{\sphinxstylestrong{\textendash{}model\_path}}
\sphinxAtStartPar
Specifies the path of the pre\sphinxhyphen{}trained model for fine\sphinxhyphen{}tuning. Default: \sphinxcode{\sphinxupquote{None}}.

\sphinxlineitem{\sphinxstylestrong{\textendash{}ol\_refit}}
\sphinxAtStartPar
For ARIMA and SARIMAX models, if specified, the model is retrained for each added observation in open\sphinxhyphen{}loop forecasts. Default: \sphinxcode{\sphinxupquote{False}}.

\end{description}


\subsection{Usage Example}
\label{\detokenize{docs/parser_arguments:usage-example}}
\sphinxAtStartPar
To run the application in training mode with an ARIMA model with validation, use:
{\color{red}\bfseries{}\textasciigrave{}\textasciigrave{}}python main\_code.py \textendash{}run\_mode training \textendash{}dataset\_path ‘/path/to/dataset.csv’ \textendash{}target\_column ‘open’ \textendash{}model\_type “ARIMA” \textendash{}scaling \textendash{}validation \textasciigrave{}\textasciigrave{}


\chapter{References}
\label{\detokenize{index:references}}
\sphinxAtStartPar
\sphinxurl{https://www.statsmodels.org/stable/examples/notebooks/generated/statespace\_forecasting.html\#Cross-validation}
\sphinxurl{https://www.statsmodels.org/stable/generated/statsmodels.tsa.ar\_model.AutoRegResults.append.html\#statsmodels.tsa.ar\_model.AutoRegResults.append}


\chapter{Indices}
\label{\detokenize{index:indices}}\begin{itemize}
\item {} 
\sphinxAtStartPar
\DUrole{xref,std,std-ref}{genindex}

\item {} 
\sphinxAtStartPar
\DUrole{xref,std,std-ref}{modindex}

\item {} 
\sphinxAtStartPar
\DUrole{xref,std,std-ref}{search}

\end{itemize}


\renewcommand{\indexname}{Python Module Index}
\begin{sphinxtheindex}
\let\bigletter\sphinxstyleindexlettergroup
\bigletter{d}
\item\relax\sphinxstyleindexentry{data\_loader}\sphinxstyleindexpageref{docs/data_loader:\detokenize{module-data_loader}}
\item\relax\sphinxstyleindexentry{data\_preprocessing}\sphinxstyleindexpageref{docs/data_preprocessing:\detokenize{module-data_preprocessing}}
\indexspace
\bigletter{m}
\item\relax\sphinxstyleindexentry{main\_code}\sphinxstyleindexpageref{docs/main_code:\detokenize{module-main_code}}
\item\relax\sphinxstyleindexentry{model\_testing}\sphinxstyleindexpageref{docs/model_testing:\detokenize{module-model_testing}}
\indexspace
\bigletter{p}
\item\relax\sphinxstyleindexentry{performance\_measurement}\sphinxstyleindexpageref{docs/performance_measurement:\detokenize{module-performance_measurement}}
\indexspace
\bigletter{t}
\item\relax\sphinxstyleindexentry{time\_series\_analysis}\sphinxstyleindexpageref{docs/time_series_analysis:\detokenize{module-time_series_analysis}}
\item\relax\sphinxstyleindexentry{training\_module}\sphinxstyleindexpageref{docs/training_module:\detokenize{module-training_module}}
\indexspace
\bigletter{u}
\item\relax\sphinxstyleindexentry{utilities}\sphinxstyleindexpageref{docs/utilities:\detokenize{module-utilities}}
\end{sphinxtheindex}

\renewcommand{\indexname}{Index}
\printindex
\end{document}